\section{Introduction}
Retailers like to maximize the purchase of certain products (or product
categories) by using personalized promotions/deals to stimulate sales, improve
customer retention or introduce people to new products or new product
categories. For example, using personalized discounts or offers to try to get
customers to buy products in new or other segments. This raises the question
whether purchase decisions can be influenced by using personalized/targeted
discounts (by using discount coupons). Since using coupons is an effective way
to provide personalized advertisement, we try to determine whether past coupon
use have significant predictive power for future purchase decisions. Therefore,
our central research question is: \textit{Can past coupon usage and past
purchase frequency predict future purchase decisions?}

Since people tend to stick to a particular product (or product category) either
by brand loyalty (retention), habit, or do not like to try things outside of
their comfort zone, using past purchase frequency can be a decent indicator of
future purchasing decisions. Moreover, using personalized coupons gives
customers an incentive to try out other products or other product categories
since they might not be even familiar with a certain product (or category) or
the discount might make them consider trying an alternative product. Therefore,
we will analyze both the effects of past purchase habits (by means of past
purchasing frequency) and the effects of using personalized coupons on the
future purchase decisions of customers.

To answer our proposed research question, we use a dataset which consists of
baskets: the week (of purchases), customers (basket of a customer), product (the product
a customer purchases) and the price spent on a certain product in euro cents.
Note that a "basket" consist of a collection of multiple products in a week
(multiple rows). Furthermore, we also have dataset which consists of the past
coupons: which also has the week (of purchase), the customer (which used the
personalized coupon), the product (which has a coupon for discount) and the
discount in euro cents.

We find that indeed, that both past purchasing decisions and (past) coupon use
have significant predictive power for the future purchasing decisions. These
findings are important for retailers (especially if they collect purchase data and
coupon usage data) because they can use both past purchase data and personalized
advertising (by means of discount coupons) to either promote products or product
categories to increase sales.

The remainder of this report is organized as follows. First, in Section 2, we
give the problem formalization (methods, models and assumptions). 
Second, in Section 3, we describe our data and do some descriptive analysis. 
Third, in Section 4, we present our approach and baselines. Fourth, in Section 5, 
we present our findings/results. Finally, in Section 6, we conclude our
analysis, answer the research question and give some recommendations.
