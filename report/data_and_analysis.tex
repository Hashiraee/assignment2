\section{Data and Descriptive Analysis}
To perform the analysis to answer our research question raised in Section 2.1,
we examine the dataset of the customers and coupons usage obtained from
professor Sebastian Gabel during the Learning From Big Data \textbf{(LFBD)} (2022-2023) course.
This section gives an insight to the datasets (included variables, descriptive
statistics and key factors). These datasets enable us to extract (or calculate) the features
(past purchase frequency and past coupon usage) we need to determine whether
future purchase decisions can be predicted.

\subsection{Features and Target Variable Construction}
First, we define the target variable (the variable of interest). Note that the
index i (i = 0, 1, \dots, 1999) corresponds to a certain customer, the index j (j = 0,
1, \dots, 249) corresponds to a certain product and the index k (k = 0, 1,
\dots, n) where n = 1378720 corresponds to an observation of our customers
dataset.

\textit{Let $y_{ij,k}$ be the probability that customer i purchases product j
for the k'th observation}.

Our target variable, the probability that customer i purchases 
product j for the k'th observation, (referred to as $y_{ij,k}$ from here on) 
is distributed between 0 and 1 since it is a probability. 
This is relevant because we are trying to predict the probability that a customer i 
will purchase a certain product j in the future.

Second, we define our first feature, the past purchase frequency.

\textit{Let $past\_purchases_{ij,k}$ be the past purchases (frequency) of a
product j for customer i for the k'th observation}.

The past purchases variable/feature is constructed by summing up the total
number of times a certain customer i purchases a certain product j and dividing
it by the total number of weeks. \textbf{Furthermore,} note that index k is not
relevant for this variable, since the only things necessary for this feature is
whether a customer i purchased a product j, how often a customer purchased that
product, and what the total number of weeks is (90 weeks for our whole dataset).

Third, we define our second feature, the (past) coupon usage.

\textit{Let $coupon\_use_{ij,k}$ be the (past) coupon use of a
product j for customer i for the k'th observation}.

This feature is constructed by using the coupons dataset and appending the
columns of the coupons dataset to the customers dataset for the corresponding
customers, since the coupons dataset have a column indicating to which customer
they were assigned/given.

\subsection{Data Description}
A total of 1378720 observations are included in the dataset provided by the LFBD
course. In Table 1, we describe the customer/baskets dataset and its
variables. In Table 2, we describe the coupons dataset and its variables. Note
that the \textbf{price} and \textbf{discount} variables are given in euro cents.

\begin{table}[H]
\parbox{.45\linewidth}{
\centering
\scalebox{0.85}
{
 \begin{tabular}{| c | c |}
    \hline
    Variable & Description \\
    \hline
    $Week$ & The week of purchases \\
    $Customer$ & One of the 2000 customers \\
    $Product$ & One of the 250 products purchased \\
    $Price$ & The amount spent on a product \\
    \hline
    \end{tabular}
}
\caption{Description of the Variables in the Customers/Baskets Dataset}
}
\hfill
\parbox{.45\linewidth}{
\centering
\scalebox{0.85}
{
    \begin{tabular}{| c | c |}
    \hline
    Variable & Description \\
    \hline
    $Week$ & The week of coupon use \\
    $Customer$ & The customer who used the coupon \\
    $Product$ & The product which is on the coupon \\
    $Discount$ & The discount a customer receives \\
    \hline
    \end{tabular}
}
\caption{Description of the Variables in the Coupons Dataset}
}
\end{table}

\subsection{Data Characteristics}
Next, we provide some data characteristics to provide an insight into the
variables presented in Section 3.2.

To understand customers' future purchase decisions, we use the past purchases
(frequency) and the past coupon usage to determine the future value of
$y_{ij,k}$. The past purchases and past coupon usage, which we will use as key factors
into understanding customers' future purchase decisions, provide a significant
amount of information on customer purchases. Furthermore, the past coupon usage
captures the impact of (personalized) advertisement which provides retailers an
insight into the effectiveness of personalized advertising using
coupons.

Next, we provide some data characteristics to provide an insight into some of the
variables discussed in Table 1 and Table 2. We examine the mean, standard
deviation (stdev), minimum and maximum. 
\textbf{Note}, we denote an "x" (a cross) for the cells that do not
make sense (for example, the mean or variance of a customer).

\begin{table}[H]
\parbox{.45\linewidth}{
\centering
\scalebox{0.85}
{
    \begin{tabular}{| c | c | c | c | c |}
    \hline
    & Week & Customer & Product & Price \\
    \hline
    $Mean$ & x & x & x & 584 \\
    \hline
    $Stdev$ & x & x & x & 97 \\
    \hline
    $Min$ & 0 & 0 & 0 & 234 \\
    \hline
    $Max$ & 89 & 1999 & 249 & 837 \\
    \hline
    \end{tabular}
}
\caption{Data Characteristics of the Variables in the Customers/Baskets Dataset}
}
\hfill
\parbox{.50\linewidth}{
\centering
\scalebox{0.85}
{
    \begin{tabular}{| c | c | c | c | c |}
    \hline
    & Week & Customer & Product & Discount \\
    \hline
    $Mean$ & x & x & x & 25 \\
    \hline
    $Stdev$ & x & x & x & 10 \\
    \hline
    $Min$ & 0 & 0 & 0 & 10 \\
    \hline
    $Max$ & 89 & 1999 & 249 & 40 \\
    \hline
    \end{tabular}
}
\caption{Data Characteristics of the Variables in the Coupons Dataset}
}
\end{table}
We observe that on average, customers spent 5.64 euros on products and got an
average discount of 25 cents through coupons on qualifying products.

\subsection{Data Preparation}
The dataset is complete and does not contain any missing observations, or any
(significant) outliers, therefore we use the dataset as is. Even if we find
outliers, we will not exclude those observations since they provide valuable
information. Furthermore, we will use the purchase data from the
Customers/Baskets dataset to construct one of our features: the past purchase
frequency. First, this will serve as a baseline/benchmark model (more details in
Section 4). Second, it will allow us to analyze the relationship between future
purchase decisions and past purchases.
