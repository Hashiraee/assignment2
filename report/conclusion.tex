\section{Conclusion}
Retailers use personalized advertisement by using coupons to try to increase
purchases, increase brand awareness or enter a new segment of a market. A
insight into the future purchases of consumers gives retailers an edge in
maximizing their profits by (for example) increasing sales. Accurate predictions
into future consumer purchases are therefore a convenient way for retailers to
determine the effects of both past purchasing habits and personalized
advertising through coupons. In order to find predictions which are as accurate
as possible, we compare a baseline model (with only past purchase frequency as
feature) against a linear regression model which uses past purchasing
data/frequency (with multiple time horizons such as monthly, weekly, quarterly
etc.).

In this research, we therefore formulate an answer to our original research
question: \textit{Can (past) coupon usage and past purchasing frequency predict
future purchase decisions?}

To answer the research question, we consider our baseline model (with only the
past purchasing frequency feature), our linear regression model (with past
purchasing frequency on multiple time horizons) and past coupon usage and our
random forest model using the same features as the linear regression model
mentioned before. We find that predictions using the Random Forest model
significantly outperform the both the baseline model and the linear regression
model in terms of cross entropy loss. Therefore, the best predictions are
obtained using the Random Forest model with the past purchasing frequency (on
multiple time horizons) and with past coupon use. The before mentioned implies
that indeed the past purchasing frequency and coupon use are significantly
import features into predicting the future purchase predictions of consumers.

Finding the significance of both using our Random Forest model, but more
importantly the past purchasing frequency and coupon use, gives potential
retailers an advantage for personalized advertising to consumers and also
obtaining/collecting purchase data in order to optimally target customers for
certain product (or categories). For retailers having access to both sales data
and coupon usage data, the predictions of future purchases will be significantly
more accurate than only having past coupon usage data. Therefore, retailers
should either try to obtain past purchasing data or try to use their own platform for sales
to be able to collect and analyze such data.

For future research, one may consider adding more data features (of customers)
to be able more effectively target certain customers. Having additional data on
consumers may result in more effective targeting. For example, the gender or age
may also provide valuable information on future purchasing decisions (purchasing
habits across different product categories). Moreover,
we can also consider stacking models (combining regression with random forest by
applying them sequentially) may result in even more accurate predictions than
only using one model by itself.
