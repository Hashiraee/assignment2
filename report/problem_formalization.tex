\section{Problem Formalization}
In this section, we introduce our problem, give a mathematical formalization of
the problem, introduce our models, and present and discuss our assumptions.
Furthermore, we define our machine learning pipeline.

\subsection{Problem Definition}
To determine the effectiveness of using past purchase frequency and past coupon
usage, we use two models which will be used to construct the predictions for
future purchases. We will then also compare the predictive performance of the
two models and choosing one to make our final purchase predictions for week 90.
The before mentioned will help us answer our main research question:
\textit{Can past coupon usage and past purchase frequency predict future purchase decisions?}

\subsection{Mathematical Formalization of the Learning Problem}
In this (sub)section, we define the learning problem in a mathematical way.
First, we define the target variable (the variable of interest). Note that the
index i (i = 0, 1, \dots, 1999) corresponds to a certain customer, the index j (j = 0,
1, \dots, 249) corresponds to a certain product and the index k (k = 0, 1,
\dots, n = 1378720) corresponds to an observation of our customers
dataset.

\textit{Let $y_{ij,k}$ be the probability of customer i purchasing product j
for the k'th observation.}

Next, we define our two selected features to predict future customer purchases,
being past purchase frequency (past purchases) and past coupon usage.

\textit{Let $past\_purchases_{ij,k}$ be the past purchase frequency of 
customer i purchasing product j for the k'th observation.} \\
\textit{Let $coupon\_use_{ij,k}$ be the (past) coupon usage of 
customer i purchasing product j for the k'th observation.}

So our $y_{ij,k}$ is a function of our two selected features being past purchase
frequency and (past) coupon usage:
\begin{equation}
    y_{ij,k} = f(past\_purchases_{ij,k}, coupon\_use_{ij,k}; \theta).
\end{equation} Where $\theta$ is a parameter vector corresponding to our 2
selected features (variables).

\subsection{Problem Approach}
In this (sub)section we briefly introduce our approach to solve our research
question defined in Subsection 2.1. We will tackle our problem by using two
models: a Linear Regression model and a Random Forest model. Both of these
models are Supervised Learning models, because our dataset(s) consist of data
which already has labels (there are 250 product categories, and we need to
classify our test data and validation data into one of those 250 categories).
Therefore, using Supervised Learning methods appears to be the most
straightforward choice.

\subsection{Assumptions}
Here we state and discuss the assumptions needed/required for using our selected
(supervised learning) models.

\subsubsection{Selected Features: Assumptions}
For our two selected features introduced in Subsection 2.2, we assume that there
is no missing data and that there are no significant outliers. We will
discuss/verify these assumptions in the next section (Section 3).

\subsubsection{Linear Regression Model: Assumptions}
To justify the use of linear regression, we need to state the seven underlying
assumptions of Ordinary Least Squares \textbf{(OLS)}. \\
\textbf{Assumption 1}, independent (uncorrelated) explanatory variables. \\
\textbf{Assumption 2}, the expected values of error terms ($\epsilon_{ij,k}$) are zero. \\
\textbf{Assumption 3}, the error terms ($\epsilon_{ij,k}$) have equal variance
(homoscedasticity). \\
\textbf{Assumption 4}, independence of the error terms. \\
\textbf{Assumption 5}, fixed and random values for the error terms. \\
\textbf{Assumption 6}, using a linear model. \\
\textbf{Assumption 7}, the error terms are normal independent identically
distributed.

\subsubsection{Random Forest Model: Assumptions}
To justify the use of the Random Forest model/algorithm, we need to verify the
following assumptions. \\
\textbf{Assumption 1}, at each step of building the individual (sub)tree
we can find the best split of data. \\
\textbf{Assumption 2}, while building a (sub)tree we do not use the whole
dataset, but a bootstrap sample. \\
\textbf{Assumption 3}, need to aggregate the individual tree outputs by
averaging. \\
However, there is no formal distributional assumption, since random forest is a
non-parametric model and can handle skewed and categorical data (like in our
case).
