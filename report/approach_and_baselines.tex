\section{Approach and Baselines}
In this section, we introduce our selected learning algorithms, present the
definition of our model in mathematical terms (if possible), and discuss our
approach and baselines. In addition, we will explain which model we selected to
construct our predictions. With the predictions we will get using our selected
model, we will be able to answer our research central question (defined in
Section 2.1). The models we selected to use for our predictions are both
supervised learning models, a linear regression model and a random forest model.

For our analysis, we do not plan to use multiple sequential model stages
(because we did not select models which require that). We
will also not stack models, but we will use two different supervised learning
models in order to compare their performance against each other to determine
which model produces the (significantly) more accurate predictions. Using
multiple models for the same target variable ($y_{ij,k}$) gives us an idea about
whether a model gives us more accurate predictions. Moreover, we will apply the
linear regression and random forest separately, and not stack models (so we
do not use multiple models to derive our target variable). Note, that
for our random forest model, all the trees are built independently. Therefore,
since there is no dependence between trees, all trees can be built in parallel.

The predictions we will make using one of our two selected models will be the
predictions for the future purchase decisions customers make using past purchase
frequency and coupon usage. This will help retailers who use personalized
coupons as a way of personalized advertising understand whether the coupons
indeed have significant effect on consumer purchasing behavior. And if retailers
have access to both the coupons they provide and the past purchasing data of the
consumers, they can target customers more effectively in order to increase brand
awareness or increase/maximize sales. Past purchase data (or frequency) is a
decent indicator for customer retention (or brand loyalty) and overall consumer
behavior.

\subsection{Data Preparation}
After analyzing our data, we look for missing data and outliers to prepare the
data for our analysis. The process for data preparation/processing 
is discussed in the data section (Section 3).

\subsection{Data splits}
Our data consist of 90 weeks. We use the first 88 weeks for training our models,
to ensure we have enough data to make our predictions as accurate as possible.
Next, we use the data of week 89 as our testing dataset. Finally, we will use the
last week (week 90) as our validation dataset.

\subsection{Model Training}
We will use the first 88 weeks to train our models to provide sufficient data
for our models. For the linear regression model, this will yield estimators
which are as 'accurate' as possible. For our random forest model using the 88
weeks of data will provide enough data to make the classifications as accurate
(low bias) as possible. We will evaluate whether our model generalizes well to
unseen data (to the general population) by using the our 1 week test data (week
89) and our 1 week validation data (week 90) so 2 weeks in total. If we find
that the performance metric we use shows us a significant improvement from our
baseline model, we can conclude that our model generalizes adequately well to
unseen data (more about the performance metric and baseline model in the
following subsections).

\subsection{Model Selection}

\subsection{Linear Regression Model}
First, we use a linear regression. Here, we define our models in terms of our 2
selected features:
\begin{equation}
    y_{ij,k} = \gamma_{0} + \gamma_{1}*past\_purchases_{ij,k} +
    \gamma_{2}*coupon\_use_{ij,k} + \epsilon_{ij,k}.
\end{equation}
Where \textit{past\_purchases}$_{ij,k}$ corresponds to the past purchase
frequency feature and 
\textit{coupon\_use}$_{ij,k}$ corresponds to the past coupon usage feature. 
Our estimator for $\gamma_{0}$ corresponds to the intercept, 
$\gamma_{1}$ captures the effect of past purchases of customers and 
$\gamma_{2}$ captures the effect of coupon usage. These estimators can be used
to construct our predictions and also give an insight to the significance of our
selected features.

\subsection{Random Forest Model}
Second, we use a Random Forest model. Since models like the Random Forest Model
are known as 'Black Box' models, we can not give an explicit mathematical
definition. However, the probability of a certain customer (i) purchasing a certain
product (j) for a certain observation (k) is still a function of the past
purchases and past coupon usage, hence can be generally represented as:
\begin{equation}
    y_{ij,k} = f(past\_purchases_{ij,k}, coupon\_use_{ij,k}; \beta).
\end{equation} Where $\beta$ is a parameter vector corresponding to our 2
selected features (variables).

A random forest classification model builds on top of decision trees, another way of 
classifying information based on characteristics. Random forests use bagging (or bootstrap aggregating) 
to build a series of uncorrelated trees and average them for a better, less variant prediction. 
The way boosted trees are grown, is dependent on the concept of information gain, 
as measured by Shannon’s entropy formula:
\begin{equation}
    H(X) = \mathbb{E}\left[ I(X) \right] = \mathbb{E}\left[ -log(p(X)) \right].
\end{equation}
Here, I(X) is defined as information contained by X and thus the expected information gain 
is defined as $-log(p(X))$, the probability of X occurring. Using this metric, the nodes of a 
decision tree nodes are grown until a certain level of information gain is reached.

Using bagging, which selects random samples with replacement, defined as $b =
\{1, 2, 3, ..., B\}$ a decision tree $T_{b}$ can be grown for each bootstrapped
sample. The model first randomly selects m number of variables out of a total of
n variables, picks the best split ratio between the m variables and splits the
node into two daughter nodes. Repeating this for all b yields an ensemble of
decision trees $\{T_{b}\}^{B}_{1}$. When it comes to making a prediction for say
y, $\hat{C}_{b}(y)$, can be defined as a majority vote between the multiplicity of decision
trees, or:
\begin{equation}
    \hat{C}_{random\_forest}^{B}(y) = \text{ majority vote }
\{\hat{C}_{b}(y)\}.
\end{equation}
Reference: (https://www.math.mcgill.ca/yyang/resources/doc/randomforest.pdf). \\
The assumptions which need to be satisfied for the Random Forest model, are
stated in Section 2.4.

\subsection{Baseline Model}
For our baseline model, we use a simple model which only consists of one
feature. The one feature it does contain is also part of our two selected
features, namely the past purchase frequency
($past\_purchase\_frequency_{ij,k}$). That serves as a decent baseline and does
have a adequate cross entropy loss value.

\subsubsection{Baseline Model: Implementation}
Our baseline model is constructed by summing up all the purchases of a certain
product j by a certain customer i and dividing the total sum by the total number
of weeks. Which gives us a baseline for every customer and product combination.

\subsubsection{Baseline Model: Relevance and Selection}
Moreover, not only is it a decent baseline, but it is an adequate measure of
customer purchase behavior. Since a significant amount of consumers tend to
either stick to a certain product (or certain category), it is by itself a good
indicator of future customer purchasing decisions. We selected this baseline
because it not only uses one of our selected features (past purchase frequency),
but it also has decent predictive power over future purchase decisions by
itself.

These baselines are not only simple to implement, relevant and decently
performing but can also be used as a benchmark for prediction performance between
our models. If past purchase data itself is not available to retailers, it makes
personalized advertising significantly harder because retailers have less
information to make relevant coupons (or interesting coupons) for consumers.
Also, the insight we can gain by comparing our baseline with our models is
relevant. Since no improvement (or an insignificant improvement) against the
baseline model implies that our model is not that useful.

\subsection{Performance Monitoring}
We monitor (or check) the performance by using (binary) cross entropy loss. The
perfect prediction(s) have a loss of 0, a lower value indicates significant
better performance in terms of (prediction) accuracy of a chosen model.

\subsection{Findings / Insights}
We will present our findings and insights in the next section (Results Section).
